% !TeX root = SAP.tex
\documentclass[12pt,a4paper]{article}

% -------------------------------
% Pakiety
% -------------------------------
\usepackage[utf8]{inputenc}
\usepackage[T1]{fontenc}
\usepackage[polish]{babel}
\usepackage{geometry}
\geometry{margin=2.5cm}
\usepackage{fancyhdr}
\usepackage{datetime2}
\usepackage{titlesec}
\usepackage{hyperref}
\usepackage{graphicx}
\usepackage{booktabs}
\usepackage{setspace}
\usepackage{lipsum}
\usepackage{xcolor}
% -------------------------------
% Styl dokumentu
% -------------------------------
\onehalfspacing
\pagestyle{fancy}
\fancyhf{}
\fancyhead[L]{\includegraphics[scale = 0.2]{output-onlinepngtools.png}} % zamiast dużego obrazka
\fancyhead[R]{\today}
\fancyfoot[C]{\thepage}
\fancyfoot[R]{}
\setlength\parindent{24pt}

% -------------------------------
% Dokument
% -------------------------------
\begin{document}
\begin{titlepage}
    \centering
    % Logo szkoły
    \includegraphics[width=0.5\textwidth]{UMB-logo.png}\par\vspace{2cm}
    
    % Tytuł
    {\Huge \bfseries Statistical Analysis Plan \par}
    \vspace{1cm}
    {\Large Retrospektywna ocena determinantów plonów kukurydzy: typu nawozu, odmiany, warunków środowiskowych i zarządczych. \par}
    \vspace{2cm}
    
    % Autorzy
    {\large Autorzy: \par}
    \vspace{0.5cm}
    Julia Czyż \par
    Urszula Skuńczyk \par
    Bartosz Zdanuczyk \par
    \vfill
    % Instytucja
    Zakład Biostatystyki i Informatyki Medycznej, Uniwersytet Medyczny w Białymstoku \par
    \vspace{1cm}
    
    % Data
    {\large \today}

\end{titlepage}
\newpage
{\color{red}{Activity log}}
\newpage
\tableofcontents
\newpage

\section{Tło}
\par Kukurydza należy do najważniejszych zbóż na świecie, stanowiąc podstawę żywienia ludzi i zwierząt oraz pełniąc istotną funkcję w wielu gałęziach przemysłu, 
takich jak produkcja biopaliw, tworzyw sztucznych czy farmaceutyków. Jej znaczenie obejmuje również wymiar ekonomiczny i kulturowy, odgrywając kluczową rolę w globalnym bezpieczeństwie żywnościowym i międzynarodowym handlu. 
Dążąc do zapewnienia stabilnych i wysokich plonów, hodowcy kukurydzy oraz naukowcy prowadzą eksperymenty oceniające wpływ genotypu, środowiska oraz praktyk zarządczych na wielkość i jakość plonów. Ze względu na silną interakcję genotypu ze środowiskiem, odmiany testuje się w wielu lokalizacjach, 
co pozwala na ocenę ich stabilności plonowania w różnorodnych warunkach klimatyczno-glebowych
\subsection{Cel}
Głównym celem analizy jest ocena wpływu zastosowanego typu nawozu (1), 
odmiany (2) oraz czynników środowiskowych i zarządczych (3) na poziom plonów kukurydzy.
\subsection{Hipotezy }
    \subsubsection{Pierwszoplanowa Hipoteza}
    Pytanie 1: Ocena wpływu zastosowanego typu nawozu na poziom plonów kukurydzy.


Hipotezy związane z pytaniem 1 badania: 

H0: Typ nawozu nie ma wpływu na poziom plonów kukurydzy . 

H1: Typ nawozu ma wpływ na poziom plonów kukurydzy. 

(porównujemy nawozy i tylko plon pod uwagę - ANOVA/kruskal-wallis) 
\section{Definicje i skróty}
Lista używanych skrótów i definicji.

\section{Populacja badania}
Opis badanej populacji, kryteria włączenia/wyłączenia.

\section{Metody statystyczne}
\subsection{Opis ogólny}
Jakie metody statystyczne będą stosowane.
\subsection{Testy hipotez}
Jakie testy i poziomy istotności.
\subsection{Analizy dodatkowe}
Analizy eksploracyjne, podgrupy.

\section{Prezentacja wyników}
Format tabel, wykresów, raportów.

\section{Załączniki}
Dodatkowe materiały, np. schematy, kody.

\end{document}
